\documentclass{beamer}

\usepackage{tikz}

\usepackage{verbatim}
\usetikzlibrary{arrows,shapes}


\begin{document}
% Declare layers
\pgfdeclarelayer{background}
\pgfsetlayers{background,main}

\begin{frame}
\frametitle{Algoritmo de Kruskal}

%% Adjacency matrix of graph
%% \  a  b  c  d  e  f  g
%% a  x  7     5
%% b  7  x  8  9  7
%% c     8  x     5
%% d  5  9     x 15  6
%% e     7  5 15  x  8  9
%% f           6  8  x 11
%% g              9  11 x

\tikzstyle{vertex}=[circle,fill=black!25,minimum size=20pt,inner sep=0pt]
\tikzstyle{selected vertex} = [vertex, fill=red!24]
\tikzstyle{edge} = [draw,thick,-]
\tikzstyle{weight} = [font=\small]
\tikzstyle{selected edge} = [draw,line width=5pt,-,red!50]
\tikzstyle{ignored edge} = [draw,line width=5pt,-,black!20]

\begin{figure}
\begin{tikzpicture}[scale=1.8, auto,swap]
    % Draw a 7,11 network
    % First we draw the vertices
    \foreach \pos/\name in {{(0,2)/s}, {(-2,1)/a}, {(2,1)/b},
                            {(0,0)/c}, {(-1,-2)/d}, {(1,-2)/e}}
        \node[vertex] (\name) at \pos {$\name$};
    % Connect vertices with edges and draw weights
    \foreach \source/ \dest /\weight in {s/a/2, s/b/9,c/a/8, c/b/7,
                                         a/d/6, d/c/3, e/d/5,
                                         c/e/4,b/e/1}
        \path[edge] (\source) -- node[weight] {$\weight$} (\dest);
    % Start animating the vertex and edge selection. 
    %\foreach \vertex / \fr in {d/1,a/2,b/4,e/5,c/6}
    %    \path<\fr-> node[selected vertex] at (\vertex) {$\vertex$};
    % For convenience we use a background layer to highlight edges
    % This way we don't have to worry about the highlighting covering
    % weight labels. 
    \begin{pgfonlayer}{background}
        \pause
        \foreach \source / \dest in {e/b,s/a,c/d,c/e,c/e,d/a}
            \path<+->[selected edge] (\source.center) -- (\dest.center);
        \foreach \source / \dest / \fr in {d/e/6}
            \path<\fr->[ignored edge] (\source.center) -- (\dest.center);
    \end{pgfonlayer}
\end{tikzpicture}
\end{figure}

\end{frame}


\begin{frame}
\frametitle{Algoritmo de Prim}

%% Adjacency matrix of graph
%% \  a  b  c  d  e  f  g
%% a  x  7     5
%% b  7  x  8  9  7
%% c     8  x     5
%% d  5  9     x 15  6
%% e     7  5 15  x  8  9
%% f           6  8  x 11
%% g              9  11 x

\tikzstyle{vertex}=[circle,fill=black!25,minimum size=20pt,inner sep=0pt]
\tikzstyle{selected vertex} = [vertex, fill=red!24]
\tikzstyle{edge} = [draw,thick,-]
\tikzstyle{weight} = [font=\small]
\tikzstyle{selected edge} = [draw,line width=5pt,-,red!50]
\tikzstyle{ignored edge} = [draw,line width=5pt,-,black!20]

\begin{figure}
\begin{tikzpicture}[scale=1.8, auto,swap]
    % Draw a 7,11 network
    % First we draw the vertices
    \foreach \pos/\name in {{(0,2)/s}, {(-2,1)/a}, {(2,1)/b},
                            {(0,0)/c}, {(-1,-2)/d}, {(1,-2)/e}}
        \node[vertex] (\name) at \pos {$\name$};
    % Connect vertices with edges and draw weights
    \foreach \source/ \dest /\weight in {s/a/2, s/b/9,c/a/8, c/b/7,
                                         a/d/6, d/c/3, e/d/5,
                                         c/e/4,b/e/1}
        \path[edge] (\source) -- node[weight] {$\weight$} (\dest);
    % Start animating the vertex and edge selection. 
    \foreach \vertex / \fr in {e/2,b/2,d/4,a/5,c/3,s/6}
        \path<\fr-> node[selected vertex] at (\vertex) {$\vertex$};
    % For convenience we use a background layer to highlight edges
    % This way we don't have to worry about the highlighting covering
    % weight labels. 
    \begin{pgfonlayer}{background}
        \pause
        \foreach \source / \dest in {e/b,c/e,c/d,d/a,a/s}
            \path<+->[selected edge] (\source.center) -- (\dest.center);
    \end{pgfonlayer}
\end{tikzpicture}
\end{figure}

\end{frame}

\end{document}
