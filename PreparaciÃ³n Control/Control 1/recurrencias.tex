\documentclass[dcc]{fcfmcourse}
\usepackage{teoria}
\usepackage[utf8x]{inputenc}
\usepackage{amsmath}
\usepackage{amsfonts,setspace}
\usepackage{listings}
\usepackage{color}
\usepackage{epstopdf}

\definecolor{pblue}{rgb}{0.13,0.13,1}
\definecolor{pgreen}{rgb}{0,0.5,0}
\definecolor{porange}{rgb}{0.9,0.5,0}
\definecolor{pgrey}{rgb}{0.46,0.45,0.48}

\lstset{language=Java,
  showspaces=false,
  showtabs=false,
  breaklines=true,
  showstringspaces=false,
  breakatwhitespace=true,
  commentstyle=\color{porange},
  keywordstyle=\color{pblue},
  stringstyle=\color{pgreen},
  basicstyle=\ttfamily,
  moredelim=[il][\textcolor{pgrey}]{$ $},
  moredelim=[is][\textcolor{pgrey}]{\%\%}{\%\%}
}

\newenvironment{codebox} {\small \ttfamily \obeylines \begingroup \setstretch{-2.4}} {\endgroup}

\title{Ecuaciones de Recurrencia}
\course[CC3001]{Algoritmos y Estructuras de Datos}
\professor{Nelson Baloian}
\professor{Patricio Poblete}
\assistant{Manuel Cáceres}
\assistant{Sebastián Ferrada}
\assistant{Sergio Peñafiel}

% Si pasas el comando usedate a la clase, la fecha aparecerá bajo la lista de auxiliares.
% Puedes usar el formato de fecha por defecto de latex (y traducirla usando babel)
% o puedes escribir lo que quieras con el comando \date.
% \date{1 de Septiembre, 2015}


\begin{document}
\maketitle

\vspace{-1ex}

\begin{itemize}
\item[P1.] El algoritmo Mergesort ordena un conjunto de tamaño n de la siguiente manera: primero se divide el conjunto en dos subconjuntos de tamaño $n/2$, luego se ordena cada subconjunto recursivamente, y finalmente se mezclan ambos subconjuntos ordenados (este último proceso toma tiempo $Cn$, para alguna constante $C$. Escriba la ecuación de recurrencia para el costo de ordenar un conjunto de tamaño n y luego resuélvala usando el Teorema Maestro. Suponga que $n$ es una potencia de $2$.

\item[\textit{solución}.] la ecuación de recurrencia de este problema es:
\begin{equation}
T(n) = 2T(n/2) + Cn
\end{equation}
Pues cada vez, el problema se divide en dos subproblemas de tamaño $n/2$ y se unen los resultados en tiempo $Cn$.
Por teorema maestro, sabemos que la solución de la ecuación es:
\begin{equation}
T(n) = \Theta(n\log(n))
\end{equation}

\item[P2] Resuelva la ecuación de recurrencia:
\begin{equation}
a_n = 8a_{n −1} − 15a_{n −2}
\end{equation}
con $a_0 = 1$, $a_1 = 1$.

\item[\textit{solución}.] Como es una ecuación homogénea de segundo orden, podemos deteminar que el polinomio característico es:
\begin{equation}
\lambda^2 - 8\lambda + 15 = 0
\end{equation}
Factorizando, nos queda:
\begin{equation}
(\lambda - 3)(\lambda - 5) = 0
\end{equation}
Entonces, las soluciones son $\lambda_1 = 3$ y $\lambda_2 = 5$. Por lo tanto, la solución de la recurrencia es de la forma:
\begin{equation}
a_n = D\lambda_1^n + E\lambda_2^n
\end{equation}
Donde $D$ y $E$ son constantes a definir. Usando las condiciones iniciales, tenemos el siguiente sistema de ecuaciones:
\begin{align*}
D + E = 1\\
3D + 5E = 1
\end{align*}
Cuyas soluciones son $D = 2$ y $E = -1$. Finalmente, la solución de la recurrencia es:
\begin{equation}
a_n = 2\cdot 3^n - 5^n
\end{equation}


\end{itemize}

\end{document}
