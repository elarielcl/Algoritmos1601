\documentclass[dcc]{fcfmcourse}
\usepackage{teoria}
\usepackage[utf8x]{inputenc}
\usepackage{amsmath}
\usepackage{amsfonts,setspace}
\usepackage{listings}
\usepackage{color}
\usepackage{epstopdf}

\definecolor{pblue}{rgb}{0.13,0.13,1}
\definecolor{pgreen}{rgb}{0,0.5,0}
\definecolor{porange}{rgb}{0.9,0.5,0}
\definecolor{pgrey}{rgb}{0.46,0.45,0.48}

\lstset{language=Java,
  showspaces=false,
  showtabs=false,
  breaklines=true,
  showstringspaces=false,
  breakatwhitespace=true,
  commentstyle=\color{porange},
  keywordstyle=\color{pblue},
  stringstyle=\color{pgreen},
  basicstyle=\ttfamily,
  moredelim=[il][\textcolor{pgrey}]{$ $},
  moredelim=[is][\textcolor{pgrey}]{\%\%}{\%\%}
}

\newenvironment{codebox} {\small \ttfamily \obeylines \begingroup \setstretch{-2.4}} {\endgroup}

\title{Ecuaciones de Recurrencia}
\course[CC3001]{Algoritmos y Estructuras de Datos}
\professor{Nelson Baloian}
\professor{Patricio Poblete}
\assistant{Manuel Cáceres}
\assistant{Sebastián Ferrada}
\assistant{Sergio Peñafiel}

% Si pasas el comando usedate a la clase, la fecha aparecerá bajo la lista de auxiliares.
% Puedes usar el formato de fecha por defecto de latex (y traducirla usando babel)
% o puedes escribir lo que quieras con el comando \date.
% \date{1 de Septiembre, 2015}


\begin{document}
\maketitle

\vspace{-1ex}

\begin{itemize}
\item[P1.] El algoritmo Mergesort ordena un conjunto de tamaño n de la siguiente manera: primero se divide el conjunto en dos subconjuntos de tamaño $n/2$, luego se ordena cada subconjunto recursivamente, y finalmente se mezclan ambos subconjuntos ordenados (este último proceso toma tiempo $Cn$, para alguna constante $C$. Escriba la ecuación de recurrencia para el costo de ordenar un conjunto de tamaño n y luego resuélvala usando el Teorema Maestro. Suponga que $n$ es una potencia de $2$.

\item[\textit{solución}.] la ecuación de recurrencia de este problema es:
\begin{equation}
T(n) = 2T(n/2) + Cn
\end{equation}
Pues cada vez, el problema se divide en dos subproblemas de tamaño $n/2$ y se unen los resultados en tiempo $Cn$.
Por teorema maestro, sabemos que la solución de la ecuación es:
\begin{equation}
T(n) = \Theta(n\log(n))
\end{equation}

\item[P2] Resuelva la ecuación de recurrencia:
\begin{equation}
a_n = 8a_{n −1} − 15a_{n −2}
\end{equation}
con $a_0 = 1$, $a_1 = 1$.

\item[\textit{solución}.] Como es una ecuación homogénea de segundo orden, podemos deteminar que el polinomio característico es:
\begin{equation}
\lambda^2 - 8\lambda + 15 = 0
\end{equation}
Factorizando, nos queda:
\begin{equation}
(\lambda - 3)(\lambda - 5) = 0
\end{equation}
Entonces, las soluciones son $\lambda_1 = 3$ y $\lambda_2 = 5$. Por lo tanto, la solución de la recurrencia es de la forma:
\begin{equation}
a_n = D\lambda_1^n + E\lambda_2^n
\end{equation}
Donde $D$ y $E$ son constantes a definir. Usando las condiciones iniciales, tenemos el siguiente sistema de ecuaciones:
\begin{align*}
D + E = 1\\
3D + 5E = 1
\end{align*}
Cuyas soluciones son $D = 2$ y $E = -1$. Finalmente, la solución de la recurrencia es:
\begin{equation}
a_n = 2\cdot 3^n - 5^n
\end{equation}


\item[P3.] 
\begin{enumerate}
\item Escriba un método booleano llamado EsJeraquico, que reciba como parámetro un pun-
tero a la raíz de un árbol binario, y retorne verdadero si se cumple la siguiente condi-
ción: para todos los nodos, el valor en su campo de información debe ser mayor o igual
que los valores almacenados en sus hijos. Si cualquier nodo tiene un valor mayor que
el de su padre, debe retornarse falso. Utilice la siguiente definición:

\begin{lstlisting}[language=Java, frame=single]
class NodoArbol{
  int info;
  NodoArbol izq;
  NodoArbol der;
}
\end{lstlisting}

\item Suponga que se tiene una lista enlazada con nodos definidos por:

\begin{lstlisting}[language=Java, frame=single]
class NodoLista{
  int info;
  NodoLista sgte;
}
\end{lstlisting}

donde la variable lista apunta al comienzo de la lista, la cual se inicia con un nodo cabe-
cera. Usted debe escribir un método llamado BuscarYMover, que reciba un parámetro
de tipo int llamado x. Este método debe buscar en la lista un nodo que contenga el valor
x. Si lo encuentra, debe adelantarlo un lugar en la lista, excepto si ya estaba en primer
lugar, en cuyo caso debe dejarlo donde está. 
\end{enumerate}

\item[\textit{solución}.] adjunta como código java

\item[P3.] 
Una persona desea ir en auto desde un punto $0$ hasta un punto $N$, siguiendo una línea recta. En todos los puntos $0, 1, \ldots , N$ hay puestos de arriendo de autos. Debido a las tarifas, es posible que para ir desde el punto $i$ hasta un punto $j$ sea más económico contratar un auto desde $i$ hasta un punto intermedio $k$ y otro desde $k$ hasta $j$. Suponiendo que se conoce el costo $a(i, j)$ de contratar un auto retirándolo en el punto $i$ y entregándolo en el punto $j$, calcule el costo óptimo $C(i, j)$ para ir desde i hasta j para todo $i, j$. Para esto, encuentre una fórmula recursiva y luego aplique tabulación (programación dinámica) para encontrar un algoritmo eficiente. Analice cuánto demora el cálculo de $C(0, N)$, en función de $N$ (sólo orden de magnitud).
\item[\textit{solución}.] Como dice el enunciado, el costo $C(i,j)$ es el mínimo entre usar una ruta directa o pasar a través de alguna de las ciudades intermedias, es decir,
\begin{equation*}
C(i,j) = min\left(\lbrace a(i,j) \rbrace \cup \lbrace C(i,k) + C(k,j); \forall k \in [i+1,\ldots,j-1] \rbrace \right)
\end{equation*}

A partir de esta ecuación de recurrencia se puede programar un procedimiento que use tabulación y entregue $C(0,N)$ al finalizar. Para hacerlo crearemos una matriz de costos $C$ que se va llenando ``diagonalmente'', siendo la diagonal principal llena de $0$s, la siguiente de los $a(i,j)$ y las siguientes calculadas en base a las anteriores según la relación de recurrencia anteriormente encontrada(ver solución en código java adjunto).\\

Un análisis informal del algoritmo dice que hay que llenar la mitad de una matriz de $n\times n$, lo que es $\mathcal{O}(n^2)$ elementos, y para cada uno de ellos se realiza un número de comparaciones proporcional a la diferencia de sus coordenadas(pues $k$ toma valores entre ellas), lo que tiene como cota superior $n$, por lo que el algoritmo es $\mathcal{O}(n^3)$.\\

\textbf{Propuesto} Demuestre que el algoritmo es $\Omega (n^3)$, para esto cuente exactamente el número de comparaciones realizadas por el algoritmo. Lo que en realidad demostrará que el algoritmo es $\Theta (n^3)$.\\ \textbf{Hint:} Note que para elementos en una misma diagonal, el número de comparaciones realizadas es el mismo.

\end{itemize}

\end{document}
