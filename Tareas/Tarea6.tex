\documentclass[dcc]{fcfmcourse}
\usepackage{teoria}
\usepackage[utf8x]{inputenc}
\usepackage[nocenter]{qtree}

\usepackage{listings}
\usepackage{color}
 
\definecolor{codegreen}{rgb}{0,0.6,0}
\definecolor{codegray}{rgb}{0.5,0.5,0.5}
\definecolor{codepurple}{rgb}{0.7,0,0.82}
\definecolor{backcolour}{rgb}{0.95,0.95,0.92}
 
\lstdefinestyle{mystyle}{
	backgroundcolor=\color{backcolour},   
    commentstyle=\color{codegreen},
    keywordstyle=\color{codepurple},
    numberstyle=\tiny\color{codegray},
    stringstyle=\color{codepurple},
    basicstyle=\footnotesize,
    breakatwhitespace=false,         
    breaklines=true,                 
    captionpos=b,                    
    keepspaces=true,                 
    numbers=left,                    
    numbersep=5pt,                  
    showspaces=false,                
    showstringspaces=false,
    showtabs=false,                  
    tabsize=2
}
 
\lstset{language=Java, style=mystyle}
\title{Tarea 6: Comparación de algoritmos de ordenación}
\course[CC3001]{Algoritmos y Estructuras de Datos}
\professor{Nelson Baloian}
\professor{Patricio Poblete}
\assistant{Manuel Cáceres}
\assistant{Sebastián Ferrada}
\assistant{Sergio Peñafiel}
% Si pasas el comando usedate a la clase, la fecha aparecerá bajo la lista de auxiliares.
% Puedes usar el formato de fecha por defecto de latex (y traducirla usando babel)
% o puedes escribir lo que quieras con el comando \date.
% \date{1 de Septiembre, 2015}


\begin{document}
\maketitle
\vspace{-2ex}
\begin{center}
Fecha de Entrega: 20 de Junio  23:59hrs \\
\end{center}


\section{Introducción}

%% Explicación de la tarea
El objetivo de esta tarea es comparar empíricamente los tiempos de ejecución de diferentes algoritmos de ordenación basados en comparación vistos en el curso.

En particular pretendemos investigar la diferencia entre:
\begin{itemize}
\item Algoritmos $\mathcal{O}(n^2)$  y $\mathcal{O}(n \log n)$ .
\item Diferentes algoritmos $\mathcal{O}(n \log n)$.
\item Diferentes variantes del algoritmo QuickSort.
\end{itemize}

Para esto utilizaremos los algoritmos InsertionSort, MergeSort y 3 variantes de QuickSort, aplicados sobre arreglos de enteros de $n$ elementos que contienen una permutación al azar de $\lbrace 1, 2, \ldots , n \rbrace$.
%% Explicación de los tres algoritmos de ordenación
\subsection*{InsertionSort}
Este algoritmo mantiene un subconjunto de elementos ordenados y va ``insertando'' los elementos fuera de este conjunto, de esta manera el algoritmo logra ordenar todos los elementos.\\
Mas concretamente el algoritmo mantiene como invariante que los primeros $i$ elementos del arreglo están ordenados entre ellos, agrega el siguiente elemento del arreglo (rompiendo eventualmente el invariante) y repara el invariante ``insertando'' el nuevo elemento.\\
El número de comparaciones del algoritmo es $\mathcal{O}(n^2)$.

\subsection*{MergeSort}
Este algoritmo separa el conjunto a ordenar en dos subconjuntos de tamaño similar, ordena estos subconjuntos por separado para luego ordenar el conjunto total en un proceso denominado ``merge'' de estos subconjuntos.\\
Mas concretamente este algoritmo divide el arreglo en 2 mitades, ordena recursivamente cada una de estas mitades y las combina haciendo un ``merge'' de estas.\\
El número de comparaciones del algoritmo es $\mathcal{O}(n \log n)$.

\subsection*{QuickSort}
Este algoritmo elige un elemento del conjunto a ordenar denominado ``pivote'', generando un subconjunto de elementos menores al pivote y otro de elementos mayores al pivote en un proceso denominado ``particionar'', para finalmente ordenar cada uno de estos subconjuntos.\\
Mas concretamente estudiaremos 3 variantes incrementales de este algoritmo que tienen un número de comparaciones de $\mathcal{O}(n \log n)$ en el caso promedio.
\subsubsection*{Método clásico}
Para ordenar el arreglo se elige un ``pivote'' al azar poniendo los elementos menores al pivote a la izquierda del arreglo y los mayores a la derecha, ordenando recursivamente cada uno de los subarreglos generados.
\subsubsection*{Mediana de tres}
Análogo al método clásico, con la excepción que el ``pivote'' se escoge como la mediana de tres elementos escogidos al azar.
\subsubsection*{Mediana de tres e InsertionSort}
Análogo al método mediana de tres, excepto que la recursión del algoritmo se detiene cuando se quiere ordenar un arreglo de tamaño menor a $M$ (en esta tarea usaremos $M=10$). Finalmente como este algoritmo no deja ordenado el arreglo, al final se aplica el algoritmo InsertionSort.

\section{Implementación}
Su trabajo consiste en implementar los cinco algoritmos de ordención : InsertionSort, MergeSort, QuickSort (método clásico), QuickSort (mediana de tres), QuickSort (Mediana de tres e InsertionSort) previamente explicados.\\

Para realizar esto se espera que entregue cinco clases que implementen la siguiente interfaz, con cada uno de los algoritmos anteriores :
\begin{center}
\begin{lstlisting}
public interface Ordenacion {
 	public void ordenar(int[] a);
}
\end{lstlisting}
\end{center}

\section{Experimentación}
Una vez implementados los algoritmos de ordenación, se le pide que compare los tiempos de ejecución de estos, para lo cual tenga en cuenta lo siguiente :
\begin{itemize}
\item Ejecute las pruebas en un solo computador y especifique en su informe las características de este que usted considere pertinentes.
\item Solo mida el tiempo del algoritmo y no el de otras operaciones (como la creación de las permutaciones), sea cuidadoso con esto pues sus mediciones se podrían ver alteradas.
\item Para los algoritmos de $\mathcal{O} (n \log n)$  se recomienda considerar $n \in \lbrace 10^7 , \ldots, 10^8\rbrace$.
\item Para InsertionSort se recomienda considerar $n \in \lbrace 5 \cdot 10^4 , \ldots, 3 \cdot 10^5\rbrace$, luego, con los datos obtenidos realice una interpolación cuadrática buscando la curva que se ajusta mejor localmente para luego evaluarla en el intervalo $[10^7, 10^8]$ (y así comparar con los tiempos obtenidos para los algoritmos de $\mathcal{O} (n \log n)$).
\item Para generar las permutaciones al azar utilice la siguiente función que dado un $n$ retorna un arreglo que contiene una permutación al azar de $\lbrace 1, 2 , \ldots, n\rbrace$:
\begin{lstlisting}
import java.util.Random;
...
static public int[] permutacion(int n) {
	 	int[] a = new int[n];
	 	for (int i = 0; i < n; ++i)
	 		a[i] = i+1;
	 	
	 	Random rnd = new Random();
	    for (int i = 0; i < n; i++) {
	        int r = i + rnd.nextInt(n-i);
	        int temp = a[i];
	        a[i] = a[r];
	        a[r] = temp;
	    }
	    return a;
}
\end{lstlisting}

\end{itemize}
Con los datos obtenidos realice gráficos que le ayuden a comparar la \textit{performance} de los diferentes algoritmos, en particular preocúpese de que sus gráficos :
\begin{itemize}
\item Muestren el comportamiento asintótico correspondiente de cada algoritmo.
\item Permitan diferenciar :
\begin{itemize}
\item InsertionSort de algún algortimo de $\mathcal{O} (n \log n)$.
\item MergeSort de QuickSort.
\item Las distintas versiones de QuickSort.
\end{itemize}
\end{itemize} 
\newpage
\section{Condiciones de Entrega}
%%Ver esto comparado a otras tareas
\begin{itemize}
    \item Esta tarea debe ser resuelta en Java.
    \item Es obligatorio la entrega de un informe en formato pdf junto con su tarea (Ver siguiente sección).
    \item Esta tarea es de carácter individual, cualquier caso de copia se evaluará con la nota mínima.
    \item No olvide subir a U-cursos todos los archivos necesarios para que su tarea funcione correctamente.
    \item Debe subir los archivos de código fuente (*.java). Los archivos compilados (*.class) no serán evaluados.
    \item Cualquier duda respecto a la tarea puede ser consultada usando el foro del curso.
    \item \textbf{NO} se aceptarán atrasos.
\end{itemize}
\section{Informe}
El informe debe describir el trabajo realizado, la solución implementada, los resultados obtenidos y las conclusiones o interpretaciones de estos. Principalmente debe ser breve, describiendo cada uno de los puntos que a continuación se indican:
\begin{itemize}
\item \textbf{Portada:} Indicando número de la tarea, fecha, autor, email, código del curso, etc.
\item \textbf{Introducción:} Descripción breve del problema y su solución.
\item \textbf{Análisis del Problema:} Exponga en detalle el problema, los supuestos que pretende ocupar, casos de borde y brevemente la metodología usada para resolverlo. Indique también sus hipótesis preliminares.
\item \textbf{Solución del Problema:} Indique claramente los pasos que siguió para llegar a la solución del problema. Muestre mediante figuras y ejemplos qué es lo que realiza su código. Evite copiar todo el código fuente en el informe, sin embargo, puede mostrar las partes relevantes de éste.
\item \textbf{Resultados y Conclusiones:} Muestre los gráficos realizados y utilícelos para ilustrar sus conclusiones. Sea detallado en explicar qué partes indican tendencias en complejidad. Sea riguroso en las visualizaciones: verifique que los datos representados se ajusten a los datos obtenidos y no a lo que ud. espera mostrar.
\end{itemize}
\end{document}