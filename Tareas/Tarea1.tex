\documentclass[dcc]{fcfmcourse}
\usepackage{teoria}
\usepackage[utf8x]{inputenc}

\title{Tarea 1: Copo de nieve de Koch}
\course[CC3001]{Algoritmos y Estructuras de Datos}
\professor{Nelson Baloian}
\professor{Patricio Poblete}
\assistant{Manuel Cáceres}
\assistant{Sebastián Ferrada}
\assistant{Sergio Peñafiel}
% Si pasas el comando usedate a la clase, la fecha aparecerá bajo la lista de auxiliares.
% Puedes usar el formato de fecha por defecto de latex (y traducirla usando babel)
% o puedes escribir lo que quieras con el comando \date.
% \date{1 de Septiembre, 2015}


\begin{document}
\maketitle
\vspace{-2ex}
\begin{center}
Fecha de Entrega: 31 de Marzo 23:59hrs
\end{center}

\begin{figure}[h!]
    \centering
    \includegraphics[scale=0.4]{copo.png}
\end{figure}

\section{Introducción}
Los fractales son figuras geométricas muy interesantes que parecen fragmentadas, tienen dimensión no entera y no pueden ser expresadas de una forma paramétrica. Además muchos de ellos tienen la propiedad de autosimilitud, es decir los fractales repiten su estructura a diferentes escalas. Este tipo de fractales se pueden construir a partir de la repetición de un patrón o regla. \\

Si bien no existe una ecuación que permita describir un fractal, desde el punto de vista computacional los fractales son simplemente la aplicación de una función de dibujo recursivamente. \\

En esta tarea se deberá implementar usando \texttt{Turtle} una visualización para el fractal ''Copo de nieve de Koch''.

\section{Explicación}

\subsection{Clases \texttt{Turtle} y \texttt{StdDraw}}
Junto con el enunciado de la tarea se incluyen las clases \texttt{Turtle} y \texttt{StdDraw}, se dará una breve explicación de ellas a continuación:

\begin{itemize}
    \item Clase \texttt{StdDraw}: Esta clase facilita el uso de Canvas y el trabajo con dibujo de figuras geométricas en java usando Swing.
    \item Clase \texttt{Turtle}: Esta clase permite el dibujo sobre un canvas usando el sistema Tortuga. Este sistema se basa en indicarle a la tortuga (clase) los movimientos y giros que se quieren realizar, a medida que la tortuga avanza una linea se traza por la trayectoria descrita. Posee un constructor que recibe 4 parámetros x0,y0,a0,w,h; que sitúa la tortuga en las coordenadas (x,y) con una orientación angular a0 en un Canvas de tamaño (w,h). Posee los métodos \texttt{goForward}, \texttt{turnLeft} y \texttt{turnRight} para avanzar en linea recta, cambiar la orientación angular a la izquierda y la derecha respectivamente.
\end{itemize}

Todos los ángulos en \texttt{Turtle} son medidos en grados sexagesimales.

\subsection{Curva de Koch y Copo de Nieve}
La curva de Koch es un fractal que se puede producir utilizando la siguiente regla:

\vspace{-1ex}
\begin{enumerate}
    \item Para cada segmento recto de largo L se divide en 3 partes iguales de lados L/3
    \item El segmento medio de esta división es reemplazado por 2 segmentos de lados L/3 que forman un ángulo de 60° entre ellos
    \item Se vuelve a aplicar el paso 1 para los nuevos segmentos generados
\end{enumerate}

\vspace{-3ex}
\begin{figure}[h!]
    \centering
    \includegraphics[scale=0.42]{koch.jpg}
\end{figure}

La figura muestra las primeras 4 iteraciones de la construcción del fractal. 

\newpage
El copo de nieve de Koch, es simplemente el trazado de 3 Curvas de Koch simultaneas.

\begin{figure}[h!]
    \centering
    \includegraphics[scale=0.42]{copo2.png}
\end{figure}

En la figura se puede notar que si se comienza con los segmentos en rojo y se convierten en curvas de Koch usando el procedimiento anterior, se puede formar el copo de nieve de Koch.

\section{Implementación}

En esta tarea usted deberá entregar un archivo \texttt{Koch.java} que implementará lo siguiente:

\begin{enumerate}
    \item Cree una variable estática \texttt{int LMIN} y asígnele un valor pequeño por ejemplo 8.

    \item Cree la función recursiva \texttt{public static void curvaDeKoch(Turtle tortuga, int largo)} que recibe una instancia de Turtle y un largo, y según de esto dibuja la curva de Koch. 
    
    Utilice como caso base el largo mínimo \texttt{LMIN}, es decir si la curva a dibujar es de largo menor a \texttt{LMIN} entonces se puede aproximar por un segmento. 
    
    Note que la posición y la orientación están dados por las variables de instancia de la tortuga.
    
    \item Cree la función \texttt{main} que deberá pedir al usuario un ángulo de rotación, y debe dibujar el copo de nieve de Koch rotado según el ángulo dado.
\end{enumerate}

\newpage
\section{Reglas}

\begin{itemize}
    \item Esta tarea debe ser resuelta en Java.
    \item Es obligatorio la entrega de un informe en formato pdf junto con su tarea (Ver siguiente sección).
    \item Esta tarea es de carácter individual, cualquier caso de copia se evaluará con la nota mínima.
    \item No olvide subir a U-cursos todos los archivos necesarios para que su tarea funcione correctamente.
    \item Debe subir los archivos de código fuente (*.java). Los archivos compilados (*.class) no serán evaluados.
    \item Cualquier duda respecto a la tarea puede ser consultada usando el foro del curso.
    \item Se aceptan atrasos descontando 0.75 puntos por cada día o fracción. (Máximo 4 días).
\end{itemize}

\section{Informe}

El informe debe describir el trabajo realizado, el código fuente desarrollado, los resultados obtenidos y las conclusiones o interpretaciones de estos. Principalmente sea conciso, describiendo cada uno de los puntos que a continuación se indican.

\begin{itemize}
    \item \textbf{Portada:} Indicando número de la tarea, fecha, autor, email, código del curso.
    \item \textbf{Introducción:} Descripción breve del problema y su solución.
    \item \textbf{Análisis del problema:} Exponga en detalle el problema, los supuestos que pretende ocupar, casos de borde y brevemente la metodología usada para resolverlo.
    \item \textbf{Solución del problema:}
    \begin{itemize}
        \item Algoritmos de solución, incluyendo toda la información y figuras que considere necesarias.
        \item Partes relevantes del código fuente
        \item Ejemplos de entradas y salidas escogidos por usted.
    \end{itemize}
    \item \textbf{Modo de uso:} explicando brevemente cualquier dato extra necesario para la compilación y ejecución de su programa.
    \item \textbf{Resultados y análisis:} Todo el análisis de los resultados, los gráficos y la discusión requerida.
\end{itemize}

\end{document}