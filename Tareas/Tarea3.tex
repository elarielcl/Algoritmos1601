\documentclass[dcc]{fcfmcourse}
\usepackage{teoria}
\usepackage[utf8x]{inputenc}

\title{Tarea 3: }
\course[CC3001]{Algoritmos y Estructuras de Datos}
\professor{Nelson Baloian}
\professor{Patricio Poblete}
\assistant{Manuel Cáceres}
\assistant{Sebastián Ferrada}
\assistant{Sergio Peñafiel}
% Si pasas el comando usedate a la clase, la fecha aparecerá bajo la lista de auxiliares.
% Puedes usar el formato de fecha por defecto de latex (y traducirla usando babel)
% o puedes escribir lo que quieras con el comando \date.
% \date{1 de Septiembre, 2015}


\begin{document}
\maketitle
\vspace{-2ex}
\begin{center}
Fecha de Entrega: <COMPLETAR>
\end{center}


\section{Introducción}

\section{Explicación}

\section{Implementación}
\section{Reglas}

\begin{itemize}
    \item Esta tarea debe ser resuelta en Java.
    \item Es obligatorio la entrega de un informe en formato pdf junto con su tarea (Ver siguiente sección).
    \item Esta tarea es de carácter individual, cualquier caso de copia se evaluará con la nota mínima.
    \item No olvide subir a U-cursos todos los archivos necesarios para que su tarea funcione correctamente.
    \item Debe subir los archivos de código fuente (*.java). Los archivos compilados (*.class) no serán evaluados.
    \item Cualquier duda respecto a la tarea puede ser consultada usando el foro del curso.
    \item Se aceptan atrasos descontando 0.75 puntos por cada día o fracción. (Máximo 4 días).
\end{itemize}

\section{Informe}

El informe debe describir el trabajo realizado, el código fuente desarrollado, los resultados obtenidos y las conclusiones o interpretaciones de estos. Principalmente sea conciso, describiendo cada uno de los puntos que a continuación se indican.

\begin{itemize}
    \item \textbf{Portada:} Indicando número de la tarea, fecha, autor, email, código del curso.
    \item \textbf{Introducción:} Descripción breve del problema y su solución.
    \item \textbf{Análisis del problema:} Exponga en detalle el problema, los supuestos que pretende ocupar, casos de borde y brevemente la metodología usada para resolverlo.
    \item \textbf{Solución del problema:}
    \begin{itemize}
        \item Algoritmos de solución, incluyendo toda la información y figuras que considere necesarias.
        \item Partes relevantes del código fuente
        \item Ejemplos de entradas y salidas escogidos por usted.
    \end{itemize}
    \item \textbf{Modo de uso:} explicando brevemente cualquier dato extra necesario para la compilación y ejecución de su programa.
    \item \textbf{Resultados y análisis:} Todo el análisis de los resultados, los gráficos y la discusión requerida.
\end{itemize}

\end{document}