\documentclass[dcc]{fcfmcourse}
\usepackage{teoria}
\usepackage[utf8x]{inputenc}
\usepackage[nocenter]{qtree}


\title{Tarea 3: Expresiones Algebraicas}
\course[CC3001]{Algoritmos y Estructuras de Datos}
\professor{Nelson Baloian}
\professor{Patricio Poblete}
\assistant{Manuel Cáceres}
\assistant{Sebastián Ferrada}
\assistant{Sergio Peñafiel}
% Si pasas el comando usedate a la clase, la fecha aparecerá bajo la lista de auxiliares.
% Puedes usar el formato de fecha por defecto de latex (y traducirla usando babel)
% o puedes escribir lo que quieras con el comando \date.
% \date{1 de Septiembre, 2015}


\begin{document}
\maketitle
\vspace{-2ex}
\begin{center}
Fecha de Entrega: 2 de Mayo  23:59hrs
\end{center}


\section{Introducción}
Esta tarea consiste en implementar un tipo de dato abstracto (TDA)
que permita representar fórmulas algebraicas, incluyendo las operaciones de
crear e imprimir  de estas expresiones.\\

Para implementarlo se le pide que utilice un ``árbol binario de expresiones algebraicas'' el cual tiene en sus hojas los operandos de la expresión y en los demás nodos los operadores.\\

A modo de ilustración la expresión ``a + ((b - d) * c) - e'' se representa con el siguiente árbol binario :\\

\begin{center}
\Tree [.-  [.+ a [.* [.- b d ] c ] ] e ]
\end{center}


\section{Implementación}
\subsection{El TDA Expresion}
El TDA se debe llamar \textbf{Expresion} y provee al usuario las siguientes
operaciones:
\begin{itemize}
\item \underline{Un constructor \texttt{Expresion(String formula)}}: construye una representación
de árbol para la expresión algebraica a partir de una fórmula
que viene contenida en un String. La fórmula viene en notación postfijo, también conocida como notación polaca reversa. En la notación
de postfijo primero se introducen los operandos y luego la operación
respectiva. Por ejemplo: la expresión “a + b” en notación de postfijo
corresponde a “a b +”; la expresión “a + ((b - d) * c) - e” en notación
postfijo corresponde a “a b d - c * + e -”. Se utiliza también un operador
“menos unario” que en notación de postfijo se representa como
“\_”. Este operador toma solo un argumento. Por ejemplo, “-(1+2)” se
representa como “1 2 + \_”.
\item \underline{\texttt{String toSimpleString()}}: genera una representación imprimible, en forma
de un String conteniendo la fórmula en notación normal (infijo), usando
paréntesis por cada operación realizada. Por ejemplo, para la fórmula “a b d - c * + e -” \texttt{toSimpleString}  entrega “((a + ((b - d) * c)) - e)”.
\item \underline{\texttt{String toString()}}: misma funcionalidad que \texttt{toSimpleString}, pero usando
el \textbf{mínimo} número de paréntesis. Por ejemplo, para la fórmula “a b d - c * + e -” \texttt{toString}  entrega “a+(b-d)*c-e”.

\end{itemize}
\subsection{Uso del TDA}
En base a estas operaciones, implemente un transformador de expresiones algebraicas postfijo$\rightarrow$infijo interactivo, que lea desde la entrada fórmulas en notación de postfijo, de a una por línea, y que responda escribiendo en la salida la misma fórmula traducida a notación de infijo con un paréntesis por operación, luego el signo igual (“=”) y a continuación  la fórmula en infijo con número mínimo de paréntesis.\\
\subsection{Consideraciones y Guías}

\begin{itemize}
\item Para simplificar el parsing de la entrada (fórmula descrita en el String formula), suponga que las variables son letras en $\left[a-z\right]$ , separados por un espacio en blanco, y las operaciones permitidas son + (suma), - (resta), * (multiplicación), / (división) y \_(menos unario). En caso de cualquier entrada que no cumpla con este formato, se debe mostrar en pantalla el error y abortar el programa (System.exit(-1)).
\item Para almacenar la expresión algebraica, se debe utilizar como estructura de datos un árbol binario. La idea es que las variables son hojas y los operadores nodos, cuyos hijos izquierdo y derecho son los respectivos operandos (en caso del operador \_ su hijo izquierdo es el operando correspondiente). El constructor debe parsear la fórmula en notación de postfijo y construir el  árbol respectivo, usando una pila.
\item Para los métodos \texttt{toSimpleString()} y \texttt{toString()}, se debe recorrer el  árbol binario y generar la fórmula en notación normal (infijo) colocando paréntesis donde corresponda.
\item Para minimizar el número de paréntesis tienen que comparar la prioridad del operador del padre
con las de los operadores de los hijos en lo distintos niveles del árbol, para que expresiones como ``a + b*c'' no sean parentizadas pero ``a* (b+c)'' si. Además, considere reglas de asociatividad de manera que expresiones como ``a+b+c'' no sean parentizadas, pero ``a-(b+c)'' si.
\end{itemize}
\newpage
\section{Condiciones de Entrega}

\begin{itemize}
    \item Esta tarea debe ser resuelta en Java.
    \item Es obligatorio la entrega de un informe en formato pdf junto con su tarea (Ver siguiente sección).
    \item Esta tarea es de carácter individual, cualquier caso de copia se evaluará con la nota mínima.
    \item No olvide subir a U-cursos todos los archivos necesarios para que su tarea funcione correctamente.
    \item Debe subir los archivos de código fuente (*.java). Los archivos compilados (*.class) no serán evaluados.
    \item Cualquier duda respecto a la tarea puede ser consultada usando el foro del curso.
    \item \textbf{NO} se aceptarán atrasos.
\end{itemize}

\section{Informe}

El informe debe describir el trabajo realizado, la solución implementada, los resultados obtenidos
y las conclusiones o interpretaciones de estos. Principalmente debe ser breve, describiendo cada uno
de los puntos que a continuación se indican:

\begin{itemize}
    \item \textbf{Portada:} Indicando número de la tarea, fecha, autor, email, código del curso, etc.
    \item \textbf{Introducción:} Descripción breve del problema y su solución.
    \item \textbf{Análisis del problema:} Exponga en detalle el problema, los supuestos que pretende ocupar, casos de borde y brevemente la metodología usada para resolverlo.
    \item \textbf{Solución del problema:} Indique claramente los pasos que siguió para llegar a la solución
del problema. Muestre mediante figuras y ejemplos qué es lo que realiza su código. Evite
copiar todo el código fuente en el informe, sin embargo, puede mostrar las partes relevantes
de éste.
\item \textbf{Modo de uso:} Explicando brevemente cualquier dato necesario para la compilación y
ejecución de su programa.
\item \textbf{Resultados:} Muestre ejemplos de entradas y salidas de su programa.
\item \textbf{Discusión:} Discuta sobre implementaciones diferentes del TDA y compárelo con el uso de árboles binarios. Proponga nuevas operaciones que se podrían realizar sobre este TDA (las operaciones implementadas aquí fueron \texttt{toSimpleString} y \texttt{toString}).
\end{itemize}

\end{document}