\documentclass[dcc,sol]{fcfmcourse}
\usepackage{teoria}
\usepackage[utf8x]{inputenc}
\usepackage{amsmath}
\usepackage{amsfonts,setspace}
\usepackage{listings}
\usepackage{hyperref}
\usepackage{color}

\definecolor{pblue}{rgb}{0.13,0.13,1}
\definecolor{pgreen}{rgb}{0,0.5,0}
\definecolor{porange}{rgb}{0.9,0.5,0}
\definecolor{pgrey}{rgb}{0.46,0.45,0.48}

\lstset{language=Java,
  showspaces=false,
  showtabs=false,
  breaklines=true,
  showstringspaces=false,
  breakatwhitespace=true,
  commentstyle=\color{porange},
  keywordstyle=\color{pblue},
  stringstyle=\color{pgreen},
  basicstyle=\ttfamily,
  moredelim=[il][\textcolor{pgrey}]{$ $},
  moredelim=[is][\textcolor{pgrey}]{\%\%}{\%\%}
}

\newenvironment{codebox} {\small \ttfamily \obeylines \begingroup \setstretch{-2.4}} {\endgroup}

% COmpletar titulo
\title{Auxiliar 11 - Grafos}
\course[CC3001]{Algoritmos y Estructuras de Datos}
\professor{Nelson Baloian}
\professor{Patricio Poblete}
\assistant{Manuel Cáceres}
\assistant{Sebastián Ferrada}
\assistant{Sergio Peñafiel}

% Si pasas el comando usedate a la clase, la fecha aparecerá bajo la lista de auxiliares.
% Puedes usar el formato de fecha por defecto de latex (y traducirla usando babel)
% o puedes escribir lo que quieras con el comando \date.
% \date{1 de Septiembre, 2015}

%% En catedras se vio :
%% Quick Sort, dos formas, mediana de tres conjuntos chicos por inserción, solo un caso se ordena 
%%recursivamente la parte mas chica y la otra iterativa

%% Problemas propuestos en reunión:
%% - Programación de las maneras de quicksort, profiling de las version
%% - Merge Sort


\begin{document}
\maketitle

\vspace{-1ex}


\begin{problems}
\problem \textbf{Árboles}

Un grafo $G$ se dice que es un árbol si posee las siguientes dos propiedades :
\begin{itemize}
    \item Es acíclico, es decir, no existen caminos en el grafo que sin repetir arcos vuelvan al nodo inicial.
    \item Es conexo, es decir, entre cualquier par de nodos en el grafo existe una camino entre ellos.
\end{itemize}
Programe la función, \texttt{esArbol} que dado un grafo representado por sus listas de adyacencia retorne si este cumple las condiciones descritas anteriormente para ser un árbol.



\problem \textbf{Ciudades conectadas}

En cierto país existen $n$ ciudades, unidas por $m$ caminos. Cada camino une exactamente 2 ciudades, y los caminos son todos bidireccionales. Además no existe un camino que una una ciudad consigo misma. Sin embargo, actualmente en el país no es posible ir desde cualquier ciudad a otra pasando por ciudades intermedias, es por esto que se propone un plan para aumentar la conectividad y que sea posible llegar desde cualquier ciudad a otra. El objetivo del problema es dado el grafo que representa las ciudades (nodos) y los caminos (arcos) como listas de adyacencia, encontrar el \textbf{mínimo} número de caminos que es necesario construir para conectar todas las ciudades.

\end{problems}



\end{document}