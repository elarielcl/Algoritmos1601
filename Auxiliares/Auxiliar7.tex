\documentclass[dcc,sol]{fcfmcourse}
\usepackage{teoria}
\usepackage[utf8x]{inputenc}
\usepackage{amsmath}
\usepackage{amsfonts,setspace}
\usepackage{listings}
\usepackage{color}

\definecolor{pblue}{rgb}{0.13,0.13,1}
\definecolor{pgreen}{rgb}{0,0.5,0}
\definecolor{porange}{rgb}{0.9,0.5,0}
\definecolor{pgrey}{rgb}{0.46,0.45,0.48}

\lstset{language=Java,
  showspaces=false,
  showtabs=false,
  breaklines=true,
  showstringspaces=false,
  breakatwhitespace=true,
  commentstyle=\color{porange},
  keywordstyle=\color{pblue},
  stringstyle=\color{pgreen},
  basicstyle=\ttfamily,
  moredelim=[il][\textcolor{pgrey}]{$ $},
  moredelim=[is][\textcolor{pgrey}]{\%\%}{\%\%}
}

\newenvironment{codebox} {\small \ttfamily \obeylines \begingroup \setstretch{-2.4}} {\endgroup}

% COmpletar titulo
\title{Auxiliar 6 - Árboles balanceados y Colas de Prioridad}
\course[CC3001]{Algoritmos y Estructuras de Datos}
\professor{Nelson Baloian}
\professor{Patricio Poblete}
\assistant{Manuel Cáceres}
\assistant{Sebastián Ferrada}
\assistant{Sergio Peñafiel}

% Si pasas el comando usedate a la clase, la fecha aparecerá bajo la lista de auxiliares.
% Puedes usar el formato de fecha por defecto de latex (y traducirla usando babel)
% o puedes escribir lo que quieras con el comando \date.
% \date{1 de Septiembre, 2015}




%%CONVERSADAS EN REUNION PARA ESTE AUXILIAR

%- Listas enlazadas
%- Listas de doble enlace
%- Invertir lista enlazada (nelson lo hizo)
%- Clonar árbol
%- Reflejar
%- Ver si dos árboles tienen el mismo recorrido en inorden(armando listas), python(?)
%- Multiplicación de matrices(nelson)
%- Saber si lista es circular con un nodo
%- Otros problemas

\begin{document}
\maketitle

\vspace{-1ex}

\begin{problems}
\section*{Árboles balanceados}
\problem Implemente el método \texttt{boolean esAVL(Arbol a)}, para esto, implemente el método \texttt{int altura(Arbol a)} y úselo para verificar la condición de AVL. ¿Cuál es la complejidad de este algoritmo?
\begin{solution}
En Arbol.java
\end{solution}
\problem Implemente el método \texttt{esAVL} de forma eficiente, calculando la altura a medida que va recorriendo el árbol. Pruebe que haciéndolo de esta manera se alcanza una complejidad $O(n)$
\begin{solution}
En Arbol.java
\end{solution}
\end{problems}
\section*{Colas de Prioridad}
\begin{problems}
\problem Un heap $d-ario$ es un tipo de ``heap'', pero en el cual los nodos no hojas tienen $d$ hijos. Proponga una implementación de la clase \texttt{DHeap} que incluya inserción y extracción del menor.
\begin{solution}
En DHeap.java
\end{solution}
\problem Analice la complejidad de la inserción en un heap $d-ario$. ¿Cómo depende de $d$? ¿Existe algún valor óptimo?

\begin{solution}
Al cambiar la cantidad de hijos de los nodos esto trae 2 efectos, por una parte la altura del árbol se ve reducida dado que para representar n nodos solo se necesita un árbol de altura $\log_d{n}$, pero cada vez que se verifiquen los elementos de un mismo nivel se deberán hacer $d$ comparaciones.

Veremos la complejidad dependiendo de $d$ para ambas operaciones:

\begin{itemize}

\item \textbf{Insertar}, un elemento nuevo se agrega al final y se compara sólo con el padre para ver si sube, en el peor caso debe subir la altura del árbol es decir $\log_d{n}$, es evidente que a medida que el d se hace más grande $\log_d{n}$ se hace más cercano a 0, luego para cualquier $d\geq n$ el costo es 1 (Sólo se compara con la raíz). \\

\item \textbf{Extraer el mínimo}, la obtención y eliminación del mínimo cuesta $O(1)$ dado que está en el tope, sin embargo volver a la forma del heap debe para cada nivel encontrar el minimo del nivel y asignarlo como el nuevo padre. Esto entonces toma tiempo $T(n) = d\log_d{n}$. \\

Para encontrar el mínimo podemos derivar con respecto a d e igualar a 0.

\begin{equation*}
    T(n) = \frac{d\ln{n}}{\ln{d}} \Rightarrow
    T'(n) = \ln{n}\frac{\ln{d}-1}{\ln^2{d}} = 0
\end{equation*}

Asumiendo que $d>1$, luego $\ln{d} \neq 0$, se puede simplificar los terminos $\ln{n}$ y $\ln^2{d}$ y llegar a:

\begin{equation*}
    \ln{d}-1=0 \Rightarrow \ln{d} = 1 \Rightarrow d = e
\end{equation*}

Lo que indica que el tamaño óptimo es $e$, pero dado que $d$ debe ser entero se concluye que el óptimo es 3 (más cercano).

\end{itemize}



\end{solution}


\end{problems}




\end{document}