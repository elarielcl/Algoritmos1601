|\documentclass[dcc]{fcfmcourse}
\usepackage{teoria}
\usepackage[utf8x]{inputenc}
\usepackage{amsfonts,setspace}
\usepackage{listings}
\usepackage{color}

\definecolor{pblue}{rgb}{0.13,0.13,1}
\definecolor{pgreen}{rgb}{0,0.5,0}
\definecolor{porange}{rgb}{0.9,0.5,0}
\definecolor{pgrey}{rgb}{0.46,0.45,0.48}

\lstset{language=Java,
  showspaces=false,
  showtabs=false,
  breaklines=true,
  showstringspaces=false,
  breakatwhitespace=true,
  commentstyle=\color{porange},
  keywordstyle=\color{pblue},
  stringstyle=\color{pgreen},
  basicstyle=\ttfamily,
  moredelim=[il][\textcolor{pgrey}]{$$},
  moredelim=[is][\textcolor{pgrey}]{\%\%}{\%\%}
}

\newenvironment{codebox} {\small \ttfamily \obeylines \begingroup \setstretch{-2.4}} {\endgroup}

\title{Auxiliar 2 - Invariantes, Diagramas de Estados y Recursión}
\course[CC3001]{Algoritmos y Estructuras de Datos}
\professor{Nelson Baloian}
\professor{Patricio Poblete}
\assistant{Manuel Cáceres}
\assistant{Sebastián Ferrada}
\assistant{Sergio Peñafiel}

% Si pasas el comando usedate a la clase, la fecha aparecerá bajo la lista de auxiliares.
% Puedes usar el formato de fecha por defecto de latex (y traducirla usando babel)
% o puedes escribir lo que quieras con el comando \date.
% \date{1 de Septiembre, 2015}


\begin{document}
\maketitle

\vspace{-1ex}

\textbf{Preliminares (Java training)}
\begin{enumerate}
\item ¿Es este código correcto? Corríjalo si no es así
\begin{lstlisting}[language=Java, frame=single]
ststic public int mayor(int x, int y ) {
    return (x+y)/2+Math.abs((x-y)/2);
} 
\end{lstlisting}

\item ¿Qué realiza el siguiente código? 
%\lstset{frameround=tttt}
\begin{lstlisting}[language=Java, frame=single]
double suma = 0, n = 0;
Scanner in = new Scanner(System.in);
while(true) {
    double x = in.nextDouble();
    if (x == 0) break;
    if (x < 1.0 || x > 7.0) continue;
    suma += x; n++;
}
System.out.println("Promedio = "+suma/n);
\end{lstlisting}
Reescriba el código anterior sin utilizar \texttt{break} ni \texttt{continue}



\item Cree la función \texttt{invertirArreglo} que recibe un arreglo de \texttt{int}s e invierte sus elementos. ¿Podemos ocupar esta misma idea para invertir \texttt{String}s?
\end{enumerate}
\vspace{5ex}
\textbf{Problemas}
\begin{problems}


\problem \textbf{Distancia máxima}. Diseñe un algoritmo que permita encontrar el par de números enteros de un arreglo $A$ con $n$ elementos ($n\geq2$) tales que su distancia es máxima. Defina adecuadamente un invariante, verifique las condiciones iniciales para que este sea verdadero, defina la condición de salida e implemente el cuerpo del algoritmo.
\newpage

\problem \textbf{Un juego con monedas}. Alicia y Roberto tienen una moneda no cargada $\left(\mathbb{P}(cara) = \mathbb{P}(sello) = \frac{1}{2}\right)$ y deciden tirarla al aire un número indeterminado de veces hasta que :
\begin{itemize}
    \item Salgan de corrido cara, cara, sello. En cuyo caso Alicia gana el juego.
    \item Salgan de corrido cara, sello, sello. En cuyo caso Roberto gana el juego.
\end{itemize}
¿Es este juego justo?
\begin{enumerate}[a)]
    \item Escriba la función \texttt{lanzarMoneda} que con igual probabilidad retorna \texttt{false}(sello) o \texttt{true}(cara).
    \item Construya el diagrama de estados correspondiente al juego entre Alicia y Roberto.
    \item Implemente (basándose en el diagrama anterior) la función \texttt{juegoConMonedas} que simula el juego descrito anteriormente retornando \texttt{false} en caso de que Alicia gane y \texttt{true} en caso de que Roberto lo haga.
    \item Ejecute $n$ veces \texttt{juegoConMonedas} y observe las frecuencias con las que gana Alicia y Roberto. ¿Es este juego justo?
\end{enumerate}
\vspace{5ex}
\problem \textbf{Inversiones}. Un mercado de inversión de acciones tiene las siguientes reglas: 
\begin{itemize}
    \item Se pueden comprar y vender acciones en cualquier día.
    \item Una persona puede comprar acciones sólo si no tiene acciones sin vender.
    \item Cada vez que se compra una acción se tiene un costo fijo de \$C (Además del precio propio de la acción ese día).
\end{itemize}  
Un experto predice el precio de las acciones para los siguientes $n$ días y su predicción está en un arreglo de enteros $P$. Cree un programa que dado la predicción P y el costo fijo C, muestre cuál es la máxima ganancia que se puede obtener en ese periodo.

\textbf{Hint:} Para un día cualquiera use recursión para evaluar las ganancias de los casos de comprar/vender o pasar ese día, escoja la mejor opción.

\textbf{Propuesto:} Modifique el programa anterior para que indique exactamente que días debe comprar y vender acciones.
\end{problems}

\end{document}