\documentclass[dcc]{fcfmcourse}
\usepackage{teoria}
\usepackage[utf8x]{inputenc}
\usepackage{amsmath}
\usepackage{amsfonts,setspace}
\usepackage{listings}
\usepackage{color}

\definecolor{pblue}{rgb}{0.13,0.13,1}
\definecolor{pgreen}{rgb}{0,0.5,0}
\definecolor{porange}{rgb}{0.9,0.5,0}
\definecolor{pgrey}{rgb}{0.46,0.45,0.48}

\lstset{language=Java,
  showspaces=false,
  showtabs=false,
  breaklines=true,
  showstringspaces=false,
  breakatwhitespace=true,
  commentstyle=\color{porange},
  keywordstyle=\color{pblue},
  stringstyle=\color{pgreen},
  basicstyle=\ttfamily,
  moredelim=[il][\textcolor{pgrey}]{$ $},
  moredelim=[is][\textcolor{pgrey}]{\%\%}{\%\%}
}

\newenvironment{codebox} {\small \ttfamily \obeylines \begingroup \setstretch{-2.4}} {\endgroup}

% COmpletar titulo
\title{Auxiliar 4 - Listas Enlazadas y Árboles Binarios}
\course[CC3001]{Algoritmos y Estructuras de Datos}
\professor{Nelson Baloian}
\professor{Patricio Poblete}
\assistant{Manuel Cáceres}
\assistant{Sebastián Ferrada}
\assistant{Sergio Peñafiel}

% Si pasas el comando usedate a la clase, la fecha aparecerá bajo la lista de auxiliares.
% Puedes usar el formato de fecha por defecto de latex (y traducirla usando babel)
% o puedes escribir lo que quieras con el comando \date.
% \date{1 de Septiembre, 2015}




%%CONVERSADAS EN REUNION PARA ESTE AUXILIAR

%- Listas enlazadas
%- Listas de doble enlace
%- Invertir lista enlazada (nelson lo hizo)
%- Clonar árbol
%- Reflejar
%- Ver si dos árboles tienen el mismo recorrido en inorden(armando listas), python(?)
%- Multiplicación de matrices(nelson)
%- Saber si lista es circular con un nodo
%- Otros problemas

\begin{document}
\maketitle

\vspace{-1ex}

\begin{problems}
%% - Juntar dos listas dadas sus nodos de inicios
%% - Eliminar todos los elementos de una lista
%% - Invertir una lista solo cambiando la dirección de sus enlaces (iterativo y recursivo)
%% - Hacer operaciones con nodo cabecera
\section*{Listas Enlazadas}
\problem \textbf{Calentamiento.} Dada una lista enlazada de enteros, se pide que:
\begin{itemize}
    \item dado un entero, cuente las ocurrencias de este en la lista
    \item dado un entero, elimine todas sus ocurrencias de la lista
    \item elimine posibles duplicados en la lista, solo deje la primera ocurrencia. (\textbf{Hint:} use una lista auxiliar)
    \item invierta la lista, solo cambiando la dirección de los enlaces (i.e., sin utilizar una lista adicional, ni insertando/eliminando elementos)
\end{itemize}

\problem \textbf{Loops en listas.} Dada la estructura de las listas enlazadas, es posible que el nodo que apunta un elemento sea un elemento anterior de la misma lista generando un ciclo o loop. Cree la función \texttt{static boolean tieneLoop(Nodo first)} que dado un nodo de una lista enlazada retorna true si la lista tiene un ciclo, calcule la complejidad del algoritmo.

\section*{Árboles Binarios}
\problem Una forma de saber si dos árboles binarios son similares, es verificar si tienen recorridos equivalentes. Para esto, se pide que implemente la función \texttt{public boolean inordenEquivalente(Nodo A, Nodo B)} que reciba dos nodos y retorne \texttt{true} si los recorridos en inorden de ambos árboles son idénticos.
Para esto:
\begin{itemize}
    \item programe una función que reciba un árbol y retorne una lista enlazada con los elementos del árbol, tras ser recorrido en inorden
    \item programe una función que reciba dos listas enlazadas y diga si las listas son iguales elemento a elemento
    \item use las partes anteriores para implementar \texttt{inordenEquivalente}
\end{itemize}

\end{problems}

\end{document}
