 \documentclass[dcc,sol]{fcfmcourse}
\usepackage{teoria}
\usepackage[utf8x]{inputenc}
\usepackage{amsmath}
\usepackage{amsfonts,setspace}
\usepackage{listings}
\usepackage{color}

\usepackage{tikz}

\usepackage{verbatim}
\usetikzlibrary{arrows,shapes}

\definecolor{pblue}{rgb}{0.13,0.13,1}
\definecolor{pgreen}{rgb}{0,0.5,0}
\definecolor{porange}{rgb}{0.9,0.5,0}
\definecolor{pgrey}{rgb}{0.46,0.45,0.48}

\lstset{language=Java,
  showspaces=false,
  showtabs=false,
  breaklines=true,
  showstringspaces=false,
  breakatwhitespace=true,
  commentstyle=\color{porange},
  keywordstyle=\color{pblue},
  stringstyle=\color{pgreen},
  basicstyle=\ttfamily,
  moredelim=[il][\textcolor{pgrey}]{$ $},
  moredelim=[is][\textcolor{pgrey}]{\%\%}{\%\%}
}

\newenvironment{codebox} {\small \ttfamily \obeylines \begingroup \setstretch{-2.4}} {\endgroup}

% COmpletar titulo
\title{Auxiliar 7 - Árboles balanceados}
\course[CC3001]{Algoritmos y Estructuras de Datos}
\professor{Nelson Baloian}
\professor{Patricio Poblete}
\assistant{Manuel Cáceres}
\assistant{Sebastián Ferrada}
\assistant{Sergio Peñafiel}

% Si pasas el comando usedate a la clase, la fecha aparecerá bajo la lista de auxiliares.
% Puedes usar el formato de fecha por defecto de latex (y traducirla usando babel)
% o puedes escribir lo que quieras con el comando \date.
% \date{1 de Septiembre, 2015}




%%CONVERSADAS EN REUNION PARA ESTE AUXILIAR

%- Listas enlazadas
%- Listas de doble enlace
%- Invertir lista enlazada (nelson lo hizo)
%- Clonar árbol
%- Reflejar
%- Ver si dos árboles tienen el mismo recorrido en inorden(armando listas), python(?)
%- Multiplicación de matrices(nelson)
%- Saber si lista es circular con un nodo
%- Otros problemas

\begin{document}
\maketitle

\vspace{-1ex}


\section*{Árboles Rojo-Negro}
Los árboles Rojo-Negro son árboles de búsqueda binaria que cumplen las siguientes condiciones:
\begin{itemize}
    \item Cada nodo, además de su valor, tiene asociado un color el cual puede ser rojo o negro.
    \item La raíz es negra.
    \item Si un nodo es rojo, todos sus hijos deben ser negros.
    \item Cualquier camino desde la raíz hasta una hoja debe contener la misma cantidad de nodos negros.
\end{itemize}
Además consideraremos en nuestra implementación de estos árboles que los nodos rojos son solo hijos izquierdos de algún nodo negro.\\

\textbf{¿Por qué?}\\
Porque de esta manera podemos relacionar cada árbol Rojo-Negro de nuestra implementación con un árbol 2-3.\\

\textbf{¿Cómo?}\\
Cada vez que aparezca la estructura :\\
\begin{center}
\pgfdeclarelayer{background}
\pgfsetlayers{background,main}
%\begin{frame}
%\frametitle{Prim's algorithm}

%% Adjacency matrix of graph
%% \  a  b  c  d  e  f  g
%% a  x  7     5
%% b  7  x  8  9  7
%% c     8  x     5
%% d  5  9     x 15  6
%% e     7  5 15  x  8  9
%% f           6  8  x 11
%% g              9  11 x

\tikzstyle{vertex}=[circle,fill=black!25,minimum size=20pt,inner sep=0pt]
\tikzstyle{selected vertex} = [vertex, fill=red!24]
\tikzstyle{edge} = [draw,thick,-]
\tikzstyle{weight} = [font=\small]
\tikzstyle{selected edge} = [draw,line width=5pt,-,red!50]
\tikzstyle{ignored edge} = [draw,line width=5pt,-,black!20]

%\begin{figure}}}}}}}}}}}}}}}}
\begin{tikzpicture}[scale=1.8, auto,swap]
    % Draw a 7,11 network
    % First we draw the vertices
    \foreach \pos/\name in {{(3,2)/D}, {(1,1)/B}, {(5,1)/E},
                            {(0,0)/A}, {(2,0)/C}}
        \node[vertex] (\name) at \pos {$\name$};
    % Connect vertices with edges and draw weights
    \foreach \source/ \dest in {D/B, D/E,B/A,B/C}
        \path[edge] (\source) -- (\dest);
    
    
    % Start animating the vertex and edge selection. 
    % Red nodes
    \foreach \vertex / \fr in {B/1}
        \path<\fr-> node[selected vertex] at (\vertex) {$\vertex$};
    % For convenience we use a background layer to highlight edges
    % This way we don't have to worry about the highlighting covering
    % weight labels. 
    %\begin{pgfonlayer}{background}
    %    \pause
    %    \foreach \source / \dest in {d/a,d/f,a/b,b/e,e/c,e/g}
    %        \path<+->[selected edge] (\source.center) -- (\dest.center);
    %    \foreach \source / \dest / \fr in {d/b/4,d/e/5,e/f/5,b/c/6,f/g/7}
    %        \path<\fr->[ignored edge] (\source.center) -- (\dest.center);
    %\end{pgfonlayer}
\end{tikzpicture}
\end{center}
\newpage
Estaremos representando el nodo :
\\
\begin{center}
\begin{tikzpicture}
\tikzstyle{bplus}=[rectangle split, rectangle split horizontal,rectangle split ignore empty parts,draw]
\tikzstyle{every node}=[bplus]
\tikzstyle{level 1}=[sibling distance=60mm]
\tikzstyle{level 2}=[sibling distance=15mm]
\node {B \nodepart{two} D} [->]
  child {node {A}} 
  child {node {C}}
  child {node {E}}
;\end{tikzpicture}
\end{center}
\\
Por otro lado, las estructuras con nodos  negros simplemente representan su correspondiente estructura 2-3 con nodos simples.\\

\textbf{¿Y esta representación de que sirve?}\\
En un árbol 2-3 los caminos desde la raíz hasta alguna hoja tienen todos el mismo número de nodos (simples o dobles). Cada nodo simple que se visite en el camino es un nodo negro en nuestro árbol y cada nodo doble que se visite corresponde a un nodo negro y uno rojo, por lo tanto en nuestra implementación de árbol Rojo-Negro se cumplirá que cualquier camino desde la raíz hasta una hoja contiene la misma cantidad de nodos negros.\\

\textbf{Estrategia de inserción:}\\
Para insertar un elemento en nuestro árbol Rojo-Negro haremos una inserción como si este fuese un árbol de Búsqueda Binaria, poniéndolo en un nodo de color rojo. \\
\begin{itemize}
\item El caso en que este se inserte como hijo izquierdo de un nodo negro no hemos roto ninguno de los invariantes de nuestro árbol.
\item Casos problemáticos:
\begin{enumerate}
\item Hijo derecho de un nodo negro
\begin{enumerate}[a)]
\item Hermano negro
\pgfdeclarelayer{background}
\pgfsetlayers{background,main}
%\begin{frame}
%\frametitle{Prim's algorithm}

%% Adjacency matrix of graph
%% \  a  b  c  d  e  f  g
%% a  x  7     5
%% b  7  x  8  9  7
%% c     8  x     5
%% d  5  9     x 15  6
%% e     7  5 15  x  8  9
%% f           6  8  x 11
%% g              9  11 x

\tikzstyle{vertex}=[circle,fill=black!25,minimum size=20pt,inner sep=0pt]
\tikzstyle{selected vertex} = [vertex, fill=red!24]
\tikzstyle{edge} = [draw,thick,-]
\tikzstyle{weight} = [font=\small]
\tikzstyle{selected edge} = [draw,line width=5pt,-,red!50]
\tikzstyle{ignored edge} = [draw,line width=5pt,-,black!20]

%\begin{figure}}}}}}}}}}}}}}}}
\begin{tikzpicture}[scale=1.8, auto,swap]
    % Draw a 7,11 network
    % First we draw the vertices
    \foreach \pos/\name in {{(1,1)/B}, {(0,0)/A}, {(2,0)/C}}
        \node[vertex] (\name) at \pos {$\name$};
    % Connect vertices with edges and draw weights
    \foreach \source/ \dest in {B/C, B/A}
        \path[edge] (\source) -- (\dest);
    
    
    % Start animating the vertex and edge selection. 
    % Red nodes
    \foreach \vertex / \fr in {C/1}
        \path<\fr-> node[selected vertex] at (\vertex) {$\vertex$};
    % For convenience we use a background layer to highlight edges
    % This way we don't have to worry about the highlighting covering
    % weight labels. 
    %\begin{pgfonlayer}{background}
    %    \pause
    %    \foreach \source / \dest in {d/a,d/f,a/b,b/e,e/c,e/g}
    %        \path<+->[selected edge] (\source.center) -- (\dest.center);
    %    \foreach \source / \dest / \fr in {d/b/4,d/e/5,e/f/5,b/c/6,f/g/7}
    %        \path<\fr->[ignored edge] (\source.center) -- (\dest.center);
    %\end{pgfonlayer}
\end{tikzpicture}
%\end{figure}

%\end{frame}
\item Hermano rojo
\pgfdeclarelayer{background}
\pgfsetlayers{background,main}
%\begin{frame}
%\frametitle{Prim's algorithm}

%% Adjacency matrix of graph
%% \  a  b  c  d  e  f  g
%% a  x  7     5
%% b  7  x  8  9  7
%% c     8  x     5
%% d  5  9     x 15  6
%% e     7  5 15  x  8  9
%% f           6  8  x 11
%% g              9  11 x

\tikzstyle{vertex}=[circle,fill=black!25,minimum size=20pt,inner sep=0pt]
\tikzstyle{selected vertex} = [vertex, fill=red!24]
\tikzstyle{edge} = [draw,thick,-]
\tikzstyle{weight} = [font=\small]
\tikzstyle{selected edge} = [draw,line width=5pt,-,red!50]
\tikzstyle{ignored edge} = [draw,line width=5pt,-,black!20]

%\begin{figure}}}}}}}}}}}}}}}}
\begin{tikzpicture}[scale=1.8, auto,swap]
    % Draw a 7,11 network
    % First we draw the vertices
    \foreach \pos/\name in {{(1,1)/B}, {(0,0)/A}, {(2,0)/C}}
        \node[vertex] (\name) at \pos {$\name$};
    % Connect vertices with edges and draw weights
    \foreach \source/ \dest in {B/C, B/A}
        \path[edge] (\source) -- (\dest);
    
    
    % Start animating the vertex and edge selection. 
    % Red nodes
    \foreach \vertex / \fr in {A/1, C/2}
        \path<\fr-> node[selected vertex] at (\vertex) {$\vertex$};
    % For convenience we use a background layer to highlight edges
    % This way we don't have to worry about the highlighting covering
    % weight labels. 
    %\begin{pgfonlayer}{background}
    %    \pause
    %    \foreach \source / \dest in {d/a,d/f,a/b,b/e,e/c,e/g}
    %        \path<+->[selected edge] (\source.center) -- (\dest.center);
    %    \foreach \source / \dest / \fr in {d/b/4,d/e/5,e/f/5,b/c/6,f/g/7}
    %        \path<\fr->[ignored edge] (\source.center) -- (\dest.center);
    %\end{pgfonlayer}
\end{tikzpicture}
%\end{figure}

%\end{frame}
\end{enumerate}
\newpage
\item Hijo izquierdo de un nodo rojo
\pgfdeclarelayer{background}
\pgfsetlayers{background,main}
%\begin{frame}
%\frametitle{Prim's algorithm}

%% Adjacency matrix of graph
%% \  a  b  c  d  e  f  g
%% a  x  7     5
%% b  7  x  8  9  7
%% c     8  x     5
%% d  5  9     x 15  6
%% e     7  5 15  x  8  9
%% f           6  8  x 11
%% g              9  11 x

\tikzstyle{vertex}=[circle,fill=black!25,minimum size=20pt,inner sep=0pt]
\tikzstyle{selected vertex} = [vertex, fill=red!24]
\tikzstyle{edge} = [draw,thick,-]
\tikzstyle{weight} = [font=\small]
\tikzstyle{selected edge} = [draw,line width=5pt,-,red!50]
\tikzstyle{ignored edge} = [draw,line width=5pt,-,black!20]

%\begin{figure}}}}}}}}}}}}}}}}
\begin{tikzpicture}[scale=1.8, auto,swap]
    % Draw a 7,11 network
    % First we draw the vertices
    \foreach \pos/\name in {{(3,2)/D}, {(1,1)/B}, {(0,0)/A}, {(2,0)/C}}
        \node[vertex] (\name) at \pos {$\name$};
    % Connect vertices with edges and draw weights
    \foreach \source/ \dest in {D/B, B/C, B/A}
        \path[edge] (\source) -- (\dest);
    
    
    % Start animating the vertex and edge selection. 
    % Red nodes
    \foreach \vertex / \fr in {A/1, B/2}
        \path<\fr-> node[selected vertex] at (\vertex) {$\vertex$};
    % For convenience we use a background layer to highlight edges
    % This way we don't have to worry about the highlighting covering
    % weight labels. 
    %\begin{pgfonlayer}{background}
    %    \pause
    %    \foreach \source / \dest in {d/a,d/f,a/b,b/e,e/c,e/g}
    %        \path<+->[selected edge] (\source.center) -- (\dest.center);
    %    \foreach \source / \dest / \fr in {d/b/4,d/e/5,e/f/5,b/c/6,f/g/7}
    %        \path<\fr->[ignored edge] (\source.center) -- (\dest.center);
    %\end{pgfonlayer}
\end{tikzpicture}
%\end{figure}

%\end{frame}
\item Hijo derecho de un nodo rojo
\pgfdeclarelayer{background}
\pgfsetlayers{background,main}
%\begin{frame}
%\frametitle{Prim's algorithm}

%% Adjacency matrix of graph
%% \  a  b  c  d  e  f  g
%% a  x  7     5
%% b  7  x  8  9  7
%% c     8  x     5
%% d  5  9     x 15  6
%% e     7  5 15  x  8  9
%% f           6  8  x 11
%% g              9  11 x

\tikzstyle{vertex}=[circle,fill=black!25,minimum size=20pt,inner sep=0pt]
\tikzstyle{selected vertex} = [vertex, fill=red!24]
\tikzstyle{edge} = [draw,thick,-]
\tikzstyle{weight} = [font=\small]
\tikzstyle{selected edge} = [draw,line width=5pt,-,red!50]
\tikzstyle{ignored edge} = [draw,line width=5pt,-,black!20]

%\begin{figure}}}}}}}}}}}}}}}}
\begin{tikzpicture}[scale=1.8, auto,swap]
    % Draw a 7,11 network
    % First we draw the vertices
    \foreach \pos/\name in {{(3,2)/D}, {(1,1)/B}, {(0,0)/A}, {(2,0)/C}}
        \node[vertex] (\name) at \pos {$\name$};
    % Connect vertices with edges and draw weights
    \foreach \source/ \dest in {D/B, B/C, B/A}
        \path[edge] (\source) -- (\dest);
    
    
    % Start animating the vertex and edge selection. 
    % Red nodes
    \foreach \vertex / \fr in {C/1, B/2}
        \path<\fr-> node[selected vertex] at (\vertex) {$\vertex$};
    % For convenience we use a background layer to highlight edges
    % This way we don't have to worry about the highlighting covering
    % weight labels. 
    %\begin{pgfonlayer}{background}
    %    \pause
    %    \foreach \source / \dest in {d/a,d/f,a/b,b/e,e/c,e/g}
    %        \path<+->[selected edge] (\source.center) -- (\dest.center);
    %    \foreach \source / \dest / \fr in {d/b/4,d/e/5,e/f/5,b/c/6,f/g/7}
    %        \path<\fr->[ignored edge] (\source.center) -- (\dest.center);
    %\end{pgfonlayer}
\end{tikzpicture}
%\end{figure}

%\end{frame}
\end{enumerate}
\end{itemize}

Estos casos se dan al insertar un elemento al final del árbol y también se van dando a medida que se van reparando errores en niveles inferiores del mismo.\\
Para reparar estos errores usaremos rotaciones e intercambio de colores y veremos que algunos casos se reducen a resolver otros.\\

\textbf{Operaciones para tratar los casos anteriores}\\
Primero, el caso 1. a) es simple de resolver, pues con una rotación simple el árbol queda como :\\
\begin{center}
\pgfdeclarelayer{background}
\pgfsetlayers{background,main}
%\begin{frame}
%\frametitle{Prim's algorithm}

%% Adjacency matrix of graph
%% \  a  b  c  d  e  f  g
%% a  x  7     5
%% b  7  x  8  9  7
%% c     8  x     5
%% d  5  9     x 15  6
%% e     7  5 15  x  8  9
%% f           6  8  x 11
%% g              9  11 x

\tikzstyle{vertex}=[circle,fill=black!25,minimum size=20pt,inner sep=0pt]
\tikzstyle{selected vertex} = [vertex, fill=red!24]
\tikzstyle{edge} = [draw,thick,-]
\tikzstyle{weight} = [font=\small]
\tikzstyle{selected edge} = [draw,line width=5pt,-,red!50]
\tikzstyle{ignored edge} = [draw,line width=5pt,-,black!20]

%\begin{figure}}}}}}}}}}}}}}}}
\begin{tikzpicture}[scale=1.8, auto,swap]
    % Draw a 7,11 network
    % First we draw the vertices
    \foreach \pos/\name in {{(1,1)/B}, {(0,0)/A}, {(2,2)/C}}
        \node[vertex] (\name) at \pos {$\name$};
    % Connect vertices with edges and draw weights
    \foreach \source/ \dest in {B/C, B/A}
        \path[edge] (\source) -- (\dest);
    
    
    % Start animating the vertex and edge selection. 
    % Red nodes
    \foreach \vertex / \fr in {B/1}
        \path<\fr-> node[selected vertex] at (\vertex) {$\vertex$};
    % For convenience we use a background layer to highlight edges
    % This way we don't have to worry about the highlighting covering
    % weight labels. 
    %\begin{pgfonlayer}{background}
    %    \pause
    %    \foreach \source / \dest in {d/a,d/f,a/b,b/e,e/c,e/g}
    %        \path<+->[selected edge] (\source.center) -- (\dest.center);
    %    \foreach \source / \dest / \fr in {d/b/4,d/e/5,e/f/5,b/c/6,f/g/7}
    %        \path<\fr->[ignored edge] (\source.center) -- (\dest.center);
    %\end{pgfonlayer}
\end{tikzpicture}
%\end{figure}

%\end{frame}
\end{center}
Y el árbol vuelve a ser Rojo-Negro a este nivel según nuestra implementación.\\

Para el caso 1. b) se realizará un intercambio de colores, de esta manera el árbol queda como :
\begin{center}
\pgfdeclarelayer{background}
\pgfsetlayers{background,main}
%\begin{frame}
%\frametitle{Prim's algorithm}

%% Adjacency matrix of graph
%% \  a  b  c  d  e  f  g
%% a  x  7     5
%% b  7  x  8  9  7
%% c     8  x     5
%% d  5  9     x 15  6
%% e     7  5 15  x  8  9
%% f           6  8  x 11
%% g              9  11 x

\tikzstyle{vertex}=[circle,fill=black!25,minimum size=20pt,inner sep=0pt]
\tikzstyle{selected vertex} = [vertex, fill=red!24]
\tikzstyle{edge} = [draw,thick,-]
\tikzstyle{weight} = [font=\small]
\tikzstyle{selected edge} = [draw,line width=5pt,-,red!50]
\tikzstyle{ignored edge} = [draw,line width=5pt,-,black!20]

%\begin{figure}}}}}}}}}}}}}}}}
\begin{tikzpicture}[scale=1.8, auto,swap]
    % Draw a 7,11 network
    % First we draw the vertices
    \foreach \pos/\name in {{(1,1)/B}, {(0,0)/A}, {(2,0)/C}}
        \node[vertex] (\name) at \pos {$\name$};
    % Connect vertices with edges and draw weights
    \foreach \source/ \dest in {B/C, B/A}
        \path[edge] (\source) -- (\dest);
    
    
    % Start animating the vertex and edge selection. 
    % Red nodes
    \foreach \vertex / \fr in {B/1}
        \path<\fr-> node[selected vertex] at (\vertex) {$\vertex$};
    % For convenience we use a background layer to highlight edges
    % This way we don't have to worry about the highlighting covering
    % weight labels. 
    %\begin{pgfonlayer}{background}
    %    \pause
    %    \foreach \source / \dest in {d/a,d/f,a/b,b/e,e/c,e/g}
    %        \path<+->[selected edge] (\source.center) -- (\dest.center);
    %    \foreach \source / \dest / \fr in {d/b/4,d/e/5,e/f/5,b/c/6,f/g/7}
    %        \path<\fr->[ignored edge] (\source.center) -- (\dest.center);
    %\end{pgfonlayer}
\end{tikzpicture}
%\end{figure}

%\end{frame}
\end{center}
Volviendo a ser árbol Rojo-Negro a este nivel según nuestra implementación.\\

Por otro lado, el caso 2. al hacer una rotación simple obtenemos: \\
\begin{center}
\pgfdeclarelayer{background}
\pgfsetlayers{background,main}
%\begin{frame}
%\frametitle{Prim's algorithm}

%% Adjacency matrix of graph
%% \  a  b  c  d  e  f  g
%% a  x  7     5
%% b  7  x  8  9  7
%% c     8  x     5
%% d  5  9     x 15  6
%% e     7  5 15  x  8  9
%% f           6  8  x 11
%% g              9  11 x

\tikzstyle{vertex}=[circle,fill=black!25,minimum size=20pt,inner sep=0pt]
\tikzstyle{selected vertex} = [vertex, fill=red!24]
\tikzstyle{edge} = [draw,thick,-]
\tikzstyle{weight} = [font=\small]
\tikzstyle{selected edge} = [draw,line width=5pt,-,red!50]
\tikzstyle{ignored edge} = [draw,line width=5pt,-,black!20]

%\begin{figure}}}}}}}}}}}}}}}}
\begin{tikzpicture}[scale=1.8, auto,swap]
    % Draw a 7,11 network
    % First we draw the vertices
    \foreach \pos/\name in {{(3,1)/D}, {(1,2)/B}, {(0,1)/A}, {(2,0)/C}}
        \node[vertex] (\name) at \pos {$\name$};
    % Connect vertices with edges and draw weights
    \foreach \source/ \dest in {D/B, D/C, B/A}
        \path[edge] (\source) -- (\dest);
    
    
    % Start animating the vertex and edge selection. 
    % Red nodes
    \foreach \vertex / \fr in {A/1, D/2}
        \path<\fr-> node[selected vertex] at (\vertex) {$\vertex$};
    % For convenience we use a background layer to highlight edges
    % This way we don't have to worry about the highlighting covering
    % weight labels. 
    %\begin{pgfonlayer}{background}
    %    \pause
    %    \foreach \source / \dest in {d/a,d/f,a/b,b/e,e/c,e/g}
    %        \path<+->[selected edge] (\source.center) -- (\dest.center);
    %    \foreach \source / \dest / \fr in {d/b/4,d/e/5,e/f/5,b/c/6,f/g/7}
    %        \path<\fr->[ignored edge] (\source.center) -- (\dest.center);
    %\end{pgfonlayer}
\end{tikzpicture}
%\end{figure}

%\end{frame}
\end{center}
El cual se reduce al caso 1. b).\\

Finalmente, si nos encontramos en el caso 3. y hacemos una rotación simple sobre ``B'' obtenemos:\\
\begin{center}
\pgfdeclarelayer{background}
\pgfsetlayers{background,main}
%\begin{frame}
%\frametitle{Prim's algorithm}

%% Adjacency matrix of graph
%% \  a  b  c  d  e  f  g
%% a  x  7     5
%% b  7  x  8  9  7
%% c     8  x     5
%% d  5  9     x 15  6
%% e     7  5 15  x  8  9
%% f           6  8  x 11
%% g              9  11 x

\tikzstyle{vertex}=[circle,fill=black!25,minimum size=20pt,inner sep=0pt]
\tikzstyle{selected vertex} = [vertex, fill=red!24]
\tikzstyle{edge} = [draw,thick,-]
\tikzstyle{weight} = [font=\small]
\tikzstyle{selected edge} = [draw,line width=5pt,-,red!50]
\tikzstyle{ignored edge} = [draw,line width=5pt,-,black!20]

%\begin{figure}}}}}}}}}}}}}}}}
\begin{tikzpicture}[scale=1.8, auto,swap]
    % Draw a 7,11 network
    % First we draw the vertices
    \foreach \pos/\name in {{(3,3)/D}, {(1,1)/B}, {(0,0)/A}, {(2,2)/C}}
        \node[vertex] (\name) at \pos {$\name$};
    % Connect vertices with edges and draw weights
    \foreach \source/ \dest in {D/C, B/C, B/A}
        \path[edge] (\source) -- (\dest);
    
    
    % Start animating the vertex and edge selection. 
    % Red nodes
    \foreach \vertex / \fr in {C/1, B/2}
        \path<\fr-> node[selected vertex] at (\vertex) {$\vertex$};
    % For convenience we use a background layer to highlight edges
    % This way we don't have to worry about the highlighting covering
    % weight labels. 
    %\begin{pgfonlayer}{background}
    %    \pause
    %    \foreach \source / \dest in {d/a,d/f,a/b,b/e,e/c,e/g}
    %        \path<+->[selected edge] (\source.center) -- (\dest.center);
    %    \foreach \source / \dest / \fr in {d/b/4,d/e/5,e/f/5,b/c/6,f/g/7}
    %        \path<\fr->[ignored edge] (\source.center) -- (\dest.center);
    %\end{pgfonlayer}
\end{tikzpicture}
%\end{figure}

%\end{frame}
\end{center}
Lo que se reduce al caso 2.
\newpage


\end{document}









































\pgfdeclarelayer{background}
\pgfsetlayers{background,main}
%\begin{frame}
%\frametitle{Prim's algorithm}

%% Adjacency matrix of graph
%% \  a  b  c  d  e  f  g
%% a  x  7     5
%% b  7  x  8  9  7
%% c     8  x     5
%% d  5  9     x 15  6
%% e     7  5 15  x  8  9
%% f           6  8  x 11
%% g              9  11 x

\tikzstyle{vertex}=[circle,fill=black!25,minimum size=20pt,inner sep=0pt]
\tikzstyle{selected vertex} = [vertex, fill=red!24]
\tikzstyle{edge} = [draw,thick,-]
\tikzstyle{weight} = [font=\small]
\tikzstyle{selected edge} = [draw,line width=5pt,-,red!50]
\tikzstyle{ignored edge} = [draw,line width=5pt,-,black!20]

%\begin{figure}}}}}}}}}}}}}}}}
\begin{tikzpicture}[scale=1.8, auto,swap]
    % Draw a 7,11 network
    % First we draw the vertices
    \foreach \pos/\name in {{(3,4)/D}, {(1,2)/B}, {(5,2)/F},
                            {(0,0)/A}, {(2,0)/C}, {(4,0)/E}, {(6,0)/G}}
        \node[vertex] (\name) at \pos {$\name$};
    % Connect vertices with edges and draw weights
    \foreach \source/ \dest in {D/B, D/F,B/A,B/C,
                                         F/E, F/G}
        \path[edge] (\source) -- (\dest);
    
    
    % Start animating the vertex and edge selection. 
    % Red nodes
    \foreach \vertex / \fr in {A/1}
        \path<\fr-> node[selected vertex] at (\vertex) {$\vertex$};
    % For convenience we use a background layer to highlight edges
    % This way we don't have to worry about the highlighting covering
    % weight labels. 
    %\begin{pgfonlayer}{background}
    %    \pause
    %    \foreach \source / \dest in {d/a,d/f,a/b,b/e,e/c,e/g}
    %        \path<+->[selected edge] (\source.center) -- (\dest.center);
    %    \foreach \source / \dest / \fr in {d/b/4,d/e/5,e/f/5,b/c/6,f/g/7}
    %        \path<\fr->[ignored edge] (\source.center) -- (\dest.center);
    %\end{pgfonlayer}
\end{tikzpicture}
%\end{figure}

%\end{frame}
