\documentclass[dcc]{fcfmcourse}
\usepackage{teoria}
\usepackage[utf8]{inputenc}
\usepackage{epstopdf}

\title[1]{Clases, Strings y Archivos}
\course[CC3001]{Algoritmos y Estructuras de Datos}
\professor{Nelson Baloian}
\professor{Patricio Poblete}
\assistant{Sebastián Ferrada}
\assistant{Sergio Peñafiel}
% Si pasas el comando usedate a la clase, la fecha aparecerá bajo la lista de auxiliares.
% Puedes usar el formato de fecha por defecto de latex (y traducirla usando babel)
% o puedes escribir lo que quieras con el comando \date.
% \date{1 de Septiembre, 2015}


\begin{document}
\maketitle

\begin{problems}
\problem Escribir la clase Punto con un constructor que reciba las coordenadas reales $x$ e $y$. La clase debe contar con los siguientes métodos:
\begin{itemize}
\item \texttt{public void moveBy(double dx, double dy)}
\item \texttt{public void moveTo(double x, double y)}
\item \texttt{public void scaleBy(double scale)}
\item \texttt{public Punto substract(Punto other)}
\item \texttt{public double module()}
\end{itemize}

\problem Una palabra es palíndrome cuando se lee igual al derecho o al revés.
\begin{enumerate}
\item Escribir una función que reciba un String y diga si es un palíndrome o no
\item Escribir una función que reciba un caracter y un String y que retorne el mismo String, pero borrando las apariciones del caracter dado. Ej \texttt{borrar('a', "casa")} retorna \texttt{"cs"}
\item Utilice las funciones anteriores en un programa que lea una frase desde el teclado y diga si es palíndrome o no. Ej: "Anita lava la tina" es palíndrome, quitando los espacios y dejando todo en minúsculas.
\end{enumerate}

\problem En el archivo \textit{notas.txt} se encuentran las notas de los alumnos de CC3001 del semestre pasado. En cada línea se encuentra el nombre del alumno y las notas de los tres controles, todo separado por el caracter ';'. Ej: \texttt{Perez Pablo;4.5;3.2;6.6}.
\begin{enumerate}
\item Escribir un programa que lea el archivo y muestre los porcentajes de alumnos con nota roja y azul
\item Escribir un programa que lea el archivo y escriba en el archivo \textit{eximidos.txt} los nombres de los alumnos que tienen promedio mayor a $5.5$
\end{enumerate}

\end{problems}
\end{document}
