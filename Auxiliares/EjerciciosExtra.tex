\documentclass[dcc]{fcfmcourse}
\usepackage{teoria}
\usepackage[utf8x]{inputenc}
\usepackage{amsfonts,setspace}

\newenvironment{codebox} {\small \ttfamily \obeylines \begingroup \setstretch{-2.4}} {\endgroup}

\title{Ejercicios extra Java}
\course[CC3001]{Algoritmos y Estructuras de Datos}

\begin{document}
\maketitle

\vspace{-5ex}
\begin{problems}
\problem Cree la clase \texttt{Fraccion} con la siguiente definición:

\begin{enumerate}
\item Cree un constructor \texttt{public Fraccion(int num, int den)}, que inicializa una fracción usando num y den como numerador y denominador.
\item Cree otro constructor\footnote{Las clases en Java pueden tener más de un constructor siempre que reciban parámetros distintos} \texttt{public Fraccion(String frac)} donde frac es un string que representa una fracción por ejemplo ''3/5'', debe descomponer este string y guardar los numeros como enteros. (Puede suponer que el string siempre será válido).
\item Cree la función \texttt{public static int mcd(int a, int b)} que calcula el mcd entre 2 números, para esto use el algoritmo de Euclides este dice que el $mcd(a,b)=mcd(b,a\%b)$ además como caso base se tiene que $mcd(a,0)=a$.
\item Cree el método \texttt{public void simplificar()} que simplifica la fracción, para esto use \texttt{mcd}
\item Cree el método \texttt{public Fraccion suma(Fraccion other)} que devuelve la fracción simplificada que representa la suma de la fracción actual con la fracción other.
\item Cree el método \texttt{public String toString()} que devuelve un string que representa la fracción por ejemplo ''4/7''.
\end{enumerate}

Utilice la clase anterior en un programa interactivo que sume n fracciones. Primero deberá pedir al usuario un numero n, y luego debera pedir que ingrese n fracciones, al final deberá mostrar la suma de las n Fracciones.

\begin{codebox}
n?
3
Fraccion 1?
1/4
Fraccion 2?
1/3
Fraccion 3?
-1/12
La suma total es: 1/2
\end{codebox}

\problem Cree una función \texttt{public static int romano(String num)} que convierte un string que representa un número romano (hasta 50) a un entero. Los números romanos tienen las siguientes reglas:
\begin{itemize}
	\item Las letras I, V, X y L se asignan a los números 1, 5, 10 y 50 respectivamente.
	\item Los letras se escriben de mayor a menor asignación numérica y se suman (por ejemplo VII = 5 + 1 + 1 = 7)
	\item Existen las siguientes excepciones: IV = 4, IX=9, y XL = 40.
\end{itemize}

\end{problems}

\end{document}